\documentclass{muratcan_cv}

% Personal Information
\setname{}{}
\setmobile{}
\setmail{}
\setposition{Frontend Developer}
\setthemecolor{blue}

% Redefine contact to show only email and mobile, removing address, LinkedIn, GitHub
\makeatletter
\renewcommand{\contact}{%
  {\small \@mail \cps \@mobile}%
}
\makeatother

\begin{document}

% Create header
\headerview
\vspace{1ex}

% Summary
\addblocktext{Summary}{
총 7년의 개발 경력(웹 프론트엔드 약 4.5년, 리눅스 앱 약 2.5년)을 보유하고 있으며, 특히 WebRTC 기반 원격 화상회의 서비스 '하이퍼미팅'의 초기 기획 및 개발부터 상용화, 운영, 기능 고도화, 성능 최적화에 이르는 전 과정에 참여하여 안정적인 서비스를 구축하고 확장한 경험이 있습니다. 고객사 및 유관 부서와 긴밀히 협업하여 여러 기술 과제를 해결했고, 사용자 피드백에 대응하여 서비스 품질을 지속적으로 개선하는 데 익숙합니다. 이러한 경험을 바탕으로 최신 웹 기술 동향에 대한 통찰력과 실질적인 적용 능력을 더욱 발전시켜, 프론트엔드를 넘어 백엔드와 시스템 전반을 이해하는 풀스택 웹 아키텍트로 성장하고자 합니다.
}

% Skills
\section{Skills}
    \newcommand{\skillone}{\createskill{FrontEnd}{TypeScript \cpshalf React \cpshalf MobX \cpshalf JavaScript \cpshalf WebRTC \cpshalf HTML/CSS}}
    \newcommand{\skilltwo}{\createskill{DevOps Tools}{Webpack \cpshalf Docker}}
    \newcommand{\skillthree}{\createskill{Collaboration Tools}{Git \cpshalf Storybook \cpshalf Figma \cpshalf Lighthouse}}
    \createskills{\skillone, \skilltwo, \skillthree}

% Education
\section{Education}
    \datedexperience{한국과학기술원(KAIST)}{2016-2017}
    \explanation{정보보호대학원 석사}
    \explanationdetail{%
      \coloredbullet\ 연구 분야: 공공 사물인터넷 환경에서의 신뢰 기반 적응형 접근 제어 시스템\par
      \coloredbullet\ 주요 성과: TrustCom (CORE A) 학회 1저자 논문 2편 게재 및 구두 발표, CCNC (CORE B) 학회 3저자 논문 1편 게재%
    }
    
    \datedexperience{아주대학교}{2012-2015}
    \explanation{소프트웨어융합학과 학사 | GPA: 3.9/4.5} 
    \explanationdetail{%
      \coloredbullet\ 주요 활동: 컴퓨터 공학 전공 (알고리즘, 자료구조, 네트워크 등), 국내 학회 논문 게재, 스타트업 인턴, 경진대회 참가 및 수상 등%
    }
    
% Experience
\section{Experience}
    \datedexperience{티맥스티베로}{2024.12-2025.02}
    \explanation{Frontend Developer | 정부과제팀 | 'IITP 음성 품질 향상 과제' 담당}
    
    \datedexperience{티맥스클라우드 / 티맥스오에스}{2020.08-2024.11}
    \explanation{Frontend Developer | 하이퍼미팅 개발팀 | '하이퍼미팅' 및 '온라인 강의 시스템' 개발}

    \datedexperience{티맥스에이앤씨}{2018.02-2020.07}
    \explanation{Linux Developer | 어플리케이션 개발팀 | TmaxOS '미디어 앱' 개발}

% Projects
\section{Projects}
    \datedexperience{IITP 음성 품질 향상 과제}{2024.06-2025.02} 
    \explanation{AI 실시간 음성 처리 모델을 자사 화상회의 서비스('하이퍼미팅') 프론트엔드에 적용}
    \explanationdetail{%
      \coloredbullet\ ONNX 기반 On-Device AI 모델 로딩 및 클라이언트 추론 기능 개발\par
      \coloredbullet\ Web Audio API와 원형 큐를 활용한 실시간 음성 스트림 처리 등으로 성능 최적화, 지연 시간을 50\% 이상 단축\par
      \coloredbullet\ 개발, 최종 보고서 작성, 시연 등에 참여하여 성공적으로 과제 완료
    }
    
    \datedexperience{하이퍼미팅}{2020.08-2024.11}
    \explanation{WebRTC 기반 웹 화상회의 서비스 '하이퍼미팅' 프론트엔드 개발 및 유지보수}
    \explanationdetail{
      \coloredbullet\ \textbf{코어 시스템 및 주요 기능 개발}\par
      \hspace*{1em}\textbullet\ Atomic Design, Container-Presenter 패턴, MobX, React Router 등을 활용한 초기 아키텍처 설계 참여\par
      \hspace*{1em}\textbullet\ 미디어 관리, 회의 녹화, 화면 공유 등 주요 기능의 프론트엔드 로직 설계 및 구현\par
      \hspace*{1em}\textbullet\ Swagger를 활용한 RESTful API 명세 기반 협업 및 Axios 기반 API 연동, Socket.IO 기반 실시간 이벤트 처리 구현 \par
      \coloredbullet\ \textbf{성능 테스트 자동화 환경 구축}\par
      \hspace*{1em}\textbullet\  주요 기능의 가용성 및 성능 점검 효율화를 위해 Node.js 다중 프로세스 기반 가상 클라이언트 시뮬레이션 도구 개발\par 
      \hspace*{1em}\textbullet\ 주요 시나리오의 자동화된 워크플로우를 구현하고 Excalidraw, Rocket.Chat 등 오픈소스를 커스터마이징하여 적용\par
      \hspace*{1em}\textbullet\ 서버 부하 테스트 시간 80\% 이상 단축, 여러 클라이언트 성능 병목 지점 발견\par
      \coloredbullet\ \textbf{서비스 성능 최적화}\par
        \hspace*{1em}\textbullet\ 네트워크 최적화를 위해 Open WebRTC Toolkit 프레임워크 내 미디어 구독 로직 리팩토링 및 WebRTC 파라미터 튜닝\par
        \hspace*{1em}\textbullet\ 코드 스플리팅, 렌더링 최적화 등을 통해 서비스 로딩 속도 및 반응성 개선\par
        \hspace*{1em}\textbullet\ 성과: 입장 관련 서버 부하 50\% 이상 감소 및 저대역폭, 저사양 환경에서의 사용성 개선
    }
    
    \datedexperience{온라인 강의 시스템}{2022.06-2024.05}
    \explanation{인사혁신처 교육 플랫폼에 `하이퍼미팅'을 기반으로 실시간 강의 및 VOD 시청 환경 구축}
    \explanationdetail{
      \coloredbullet\ \textbf{강의 환경 맞춤 기능 개발}\par
      \hspace*{1em}\textbullet\ 고객사와 직접 소통하여 요구사항 정의, Role/Authorization, 서버 녹화, 학생 권한 제어 등 강의 특화 기능 및 관련 UI/UX 개발\par
      \hspace*{1em}\textbullet\ OWASP ZAP 기반 정기 취약점 점검 및 웹 보안 강화, 타사 콘텐츠 유출 방지 솔루션 연동 등\par
      \coloredbullet\ \textbf{VOD 모드 개발}\par
      \hspace*{1em}\textbullet\ HLS 기반 VOD 스트리밍 기능 및 관련 UX/모듈 개발\par
      \hspace*{1em}\textbullet\ 필요 REST API 설계 및 구현, 코드 스플리팅을 통한 모드별 성능 최적화
    }
    
    \datedexperience{TmaxOS 미디어 앱}{2018.02-2020.07}
    \explanation{Linux 기반 자사 OS 내 시스템 GUI 애플리케이션 개발} 
    \explanationdetail{%
      \coloredbullet\ FFmpeg 기반 OS 내장 비디오 플레이어 및 제어판 내 오디오 관리 기능 개발\par
      \coloredbullet\ MVC에서 MVVM으로의 리팩토링 및 미디어 프레임워크 모듈화 등을 통해 코드 관리 효율성과 유지보수성 향상
    }

    

\end{document}
